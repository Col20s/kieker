%%%%%%%%%%%%%%%%%%%%%%%%%%%%%%%%%%%%%%%%%
% Trace Analysis and Monitoring
%
% $Date$
% $Rev$:
% $Author$

\chapter{\KiekerTraceAnalysis{} Tool}\label{chap:aspectJ}

\KiekerTraceAnalysis{} implements the special feature of \Kieker{} allowing to %
monitor, analyze, and visualize (distributed) traces of method executions and %
corresponding timing information. For this purpose, it includes monitoring probes employing %
AspectJ~\cite{AspectJ-WebSite}, Java~EE Servlet~\cite{JavaServletTechnology-WebSite}, %
Spring~\cite{Spring-WebSite}, and Apache~CXF~\cite{CXF-WebSite} technology. %
Moreover, it allows to reconstruct and visualize architectural models of the %
monitored systems, e.g., as sequence and dependency diagrams. %

Section~\ref{chap:example} already introduced parts of the monitoring record %
type \class{OperationExecutionRecord}. \KiekerTraceAnalysis{} uses this record %
type to represent monitored executions and associated trace and session information. %
Figure~\ref{fig:OperationExecutionRecordClassDiagramComplete} shows a class diagram %
with all attributes of the record type \class{OperationExecutionRecord}. %
The attributes \method{className}, \method{operationName}, %
\method{tin}, and \method{tout} have been introduced before. %
The attributes \method{traceId} and \method{sessionId} are used to store %
trace and session information; \method{eoi} and \method{ess} contain control-flow %
information needed to reconstruct traces from monitoring data. %
For details on this, please refer to our technical %
report~\cite{vanHoornRohrHasselbringWallerEhlersFreyKieselhorst2009TRContinuousMonitoringOfSoftwareServicesDesignAndApplicationOfTheKiekerFramework}.

\begin{figure}[hb]\centering
\includegraphics[scale=0.8]{images/kieker_OperationExecutionRecord-complete-modified}%
\caption{The class diagram of the operation execution record}
\label{fig:OperationExecutionRecordClassDiagramComplete}
\end{figure}

\enlargethispage{1cm}

\noindent Section~\ref{sec:traceMonitoring} describes how to instrument Java %
applications for monitoring trace information. %
It presents the technology-specific probes provided by \Kieker{} for this %
purpose---with a focus on AspectJ. %
Additional technology-specific probes can be implemented based on the existing %
probes. %
Section~\ref{sec:traceAnalysisTool} presents the %
tool which can be used to analyze and visualize the recorded trace %
data.  Examples for the available analysis and visualization outputs %
provided by \KiekerTraceAnalysis{} are presented in %
Section~\ref{sec:traceAnalysisExamples}.

\section{Monitoring Trace Information}\label{sec:traceMonitoring}

The following Sections describe how to use the monitoring probes based on %
AspectJ (Section~\ref{sec:traceAnalysis:instr:AspectJ}), %
the Java Servlet~API (Section~\ref{sec:traceAnalysis:instr:servlet}), %
the Spring Framework (Section~\ref{sec:traceAnalysis:instr:spring}), and %
Apache~CXF (Section~\ref{sec:traceAnalysis:instr:cxf}) provided %
by \Kieker{}. %

\subsection{AspectJ-Based Instrumentation}\label{sec:traceAnalysis:instr:AspectJ}

AspectJ~\cite{AspectJ-WebSite} allows to weave code into the byte code of %
Java applications and libraries without requiring manual modifications of the %
source code. %
\Kieker{} includes the AspectJ-based monitoring probes %
\class{OperationExecutionAspectAnnotation}, %
\class{OperationExecutionAspectAnnotationServlet}, %
\class{OperationExecutionAspectFull}, and %
\class{OperationExecutionAspectFullServlet} %
which can be woven into Java applications at compile time and load time. %
These probes monitor method executions and corresponding %
trace and timing information. The probes with the postfix \class{Servlet} %
additionally store a session identifier within the \class{OperationExecutionRecord}. %
When the probes with name element \class{Annotation} are used, %
methods to be monitored must be annotated by the \Kieker{} %
annotation \class{@OperationExecutionMonitoringProbe}. %
This section demonstrates how to use the AspectJ-based probes to monitor %
traces based on the Bookstore application from Chapter~\ref{chap:example}. %

% \enlargethispage{1.0cm}

\NOTIFYBOX{The Java sources of the example presented in %
this section, as well as a pre-compiled binary, can be found in the %
\file{\aspectJBookstoreApplicationReleaseDirDistro{}/} directory of the %
binary release.}

% \

% This chapter will show in Section~\ref{sec:aspectJ:annotation} how
% to use AspectJ to mark methods to be monitored with a simple annotation
% in order to avoid the manual monitoring as seen in Chapter~\ref{chap:example}
% and \ref{chap:componentsMonitoring}. Once the methods are marked, the AspectJ-Weaver-Agent
% will surround the calls with the necessary code during runtime, similar
% to the manually inserted instrumentation code used in Section~\ref{sec:example:monitoring}.
% An alternative solution will be shown as well in Section~\ref{sec:aspectJ:fullweaving}, %
% where the methods to be instrumented are specified using an external configuration file %
% without requiring source code modifications. Both solutions
% can be used to reconstruct architectural views and to perform trace
% analyses. The result of both will be diagrams similar to sFigure~\ref{fig:bookstore:classAndSequenceDiagrams}.

% The idea of weaving the monitoring-code into the ``plain'' code
% during compile-time seems to suggest itself, but
% in this chapter it
% is only shown how to perform the so called load-time-weaving - the
% weaving during runtime.%, which is more flexible than the compile-time-weaving.

\begin{figure}[H]
\begin{graybox}
\dirtree{%
.1 \DirInDirTree{examples/}. %\DTcomment{The root directory of the project}.
.2 \DirInDirTree{userguide/}.
.3 \DirInDirTree{ch5--trace-monitoring-aspectj/}.
.4 \DirInDirTree{build/}\DTcomment{Directory for the Java class files}.
.5 \DirInDirTree{libs/}.
.6 BookstoreApplication.jar.
.4 \DirInDirTree{gradle/}.
.5 \DirInDirTree{wrapper/}\DTcomment{Directory for the gradle wrapper}.
.6 \ldots.
.4 \DirInDirTree{lib/} \DTcomment{Directory for the needed libraries}.
.5 \newFileDirInDirTree{\mainJarWeaver}.
.4 \DirInDirTree{src/}\DTcomment{Directory for the source code files}.
.5 \DirInDirTree{../ch5bookstore/}.
.6 Bookstore.java.
.6 BookstoreHostnameRewriter.java.
.6 BookstoreStarter.java.
.6 Catalog.java.
.6 CRM.java.
.4 \DirInDirTree{src-resources/}.
.5 \DirInDirTree{META-INF/}\DTcomment{Directory for the configuration files}.
.6 aop.xml.
.6 aop-event.xml.
.6 aop-full.xml.
.6 kieker.monitoring.adaptiveMonitoring.conf.
.6 \kiekerMonitoringProperties{}. 
.3 build.gradle.
.3 gradlew.
.3 gradlew.bat.
.3 README.txt.
}
\end{graybox}

\caption{The new directory structure of the Bookstore application}
\label{fig:bookstoreAOP:dirStructure}
\end{figure}

Figure~\ref{fig:bookstoreAOP:dirStructure} shows the directory used by the example of this section. %
The jar-file \file{\mainJarWeaver} already includes the \textit{AspectJ weaver}, %
which is registered with the JVM and weaves the monitoring instrumentation into %
the Java classes. It will be configured based on the configuration file %
\file{\file{\aopConfigFile}}, for which a working sample file is provided in the %
example's \dir{META-INF/} directory. Instead of registering the \file{\mainJarWeaver} %
as an agent to the JVM, the \file{\aspectJWeaverJar} can be used. In this case, %
the \file{\mainJar} needs to be added to the classpath.

\pagebreak

Once the necessary files have been copied to the example directory, the source code can be instrumented with the annotation
\class{OperationExecutionMonitoringProbe}. Listing~\ref{lst:BookstoreAspectJ} shows how the annotation is used.

\setJavaCodeListing
\lstinputlisting[caption=Bookstore.java, label=lst:BookstoreAspectJ,firstline=21,firstnumber=21]{\aspectJBookstoreApplicationDir/src/kieker/examples/userguide/ch5bookstore/Bookstore.java}

\noindent As a first example, each method of the Bookstore application will be annotated. The annotation can be used to instrument all methods except for constructors.

The \file{\aopConfigFile} file has to be modified to specify the classes to be considered for instrumentation by the AspectJ weaver. Listing~\ref{lst:aopConfigFileAnnotations} shows the modified configuration file.

\enlargethispage{1cm}
\setXMLListing
\lstinputlisting[caption=aop.xml, label=lst:aopConfigFileAnnotations]{\aspectJBookstoreApplicationDir/src-resources/META-INF/aop.xml}

\noindent Line~5 tells the AspectJ weaver to consider all classes inside the example package. %
AspectJ allows to use wild-cards for the definition of classes to %
include---e.g., \lstinline$<include within="bookstoreTracing.Bookstore*"/>$ to weave all %
classes with the prefix \class{Bookstore} located in a package \class{bookstoreTracing}.

Line~9 specifies the aspect to be woven into the classes. In this case, the \Kieker{} %
probe \class{OperationExecutionAspectAnnotation} is used. It requires that %
methods intended to be instrumented are annotated by %
\lstinline[language=Java]{@OperationExecutionMonitoringProbe}, as mentioned before.

Listings~\ref{lst:traceAnalysisCompileRunExample1} and %
\ref{lst:traceAnalysisCompileRunExample1Win} show how to compile and run the annotated %
Bookstore application. The \file{\aopConfigFile} must be located in a %
\dir{META-INF/} directory in the classpath---in this case the \dir{build/} directory. %
The AspectJ weaver has to be loaded as a so-called Java-agent. It weaves the %
monitoring aspect into the byte code of the Bookstore application. %
Additionally, a \file{\kiekerMonitoringProperties{}} is copied to the \dir{META-INF/} directory. %
This configuration file may be adjusted as desired (see Section~\ref{sec:monitoring:configuration}).

\


\setBashListing
\input{ch5-trace-analysis_Compile_Run_Example_1.inc}
\input{ch5-trace-analysis_Compile_Run_Example_1.inc-win}

\noindent After a complete run of the application, the monitoring files should appear in %
the same way as mentioned in Section~\ref{sec:example:monitoring} including the %
additional trace information. An example log of a complete run can be found in %
Appendix~\ref{sec:appendix:exampleConsoleOutputs:aspectJExample}.

\paragraph*{Instrumentation without annotations}%\label{sec:aspectJ:fullweaving}

AspectJ-based instrumentation without using annotations is quite simple. It is %
only necessary to modify the file \file{\aopConfigFile{}}, as shown %
in Listing~\ref{lst:aopConfigFileFull}.
\pagebreak
\setXMLListing
\lstinputlisting[caption=aop.xml, label=lst:aopConfigFileFull]{\aspectJBookstoreApplicationDir/src-resources/META-INF/aop-full.xml}

\noindent The alternative aspect \class{OperationExecutionAspectFull} is being %
activated in line~9. As indicated by its name, this aspect makes sure that every %
method within the included classes/packages will be instrumented and monitored. %
% The exact behavior can be controlled very exactly by using appropriate includes and excludes within the weaver-part of the configuration file. %
% For example,
Listing \ref{lst:aopConfigFileFull} demonstrates how to limit the %
instrumented methods to those of the class \class{BookstoreStarter}.

The commands shown in the Listings~\ref{lst:traceAnalysisCompileRunExample1} and %
\ref{lst:traceAnalysisCompileRunExample1Win} can again be used to compile and execute %
the example. Note that the annotations within the source code have no effect %
when using this aspect.

\

\WARNBOX{When using a custom aspect, it can be necessary to specify its %
classname in the \lstinline{include} directives of the \aopConfigFile{}.}

\subsection{Servlet Filters}\label{sec:traceAnalysis:instr:servlet}

The Java Servlet API~\cite{JavaServletTechnology-WebSite} includes the %
\class{javax.servlet.Filter} interface. It can be used to implement %
interceptors for incoming HTTP requests. %
\Kieker{} includes the probe %
\class{SessionAndTraceRegistrationFilter} which implements the %
\class{javax.servlet.Filter} interface. %
It initializes the session and trace information for incoming requests. %
If desired, it additionally creates an \class{OperationExecutionRecord} for each %
invocation of the filter and passes it to the \class{MonitoringController}.

% \enlargethispage{1.5cm}

Listing~\ref{lst:OperationExecutionRegistrationAndLoggingFilterInWebXML} %
demonstrates how to integrate the \class{SessionAndTraceRegistrationFilter} %
in the \file{web.xml} file of a web application.

The Java~EE Servlet container example described in Appendix~\ref{appendix:JavaEEServletExample} employs the %
\class{SessionAndTraceRegistrationFilter}.

\pagebreak

\setXMLListing
\lstinputlisting[firstline=50,lastline=61,firstnumber=50,%
caption=\class{SessionAndTraceRegistrationFilter} in a \file{web.xml} file,%
label=lst:OperationExecutionRegistrationAndLoggingFilterInWebXML]%
{\JavaEEServletExampleDir/jetty/webapps/jpetstore/WEB-INF/web.xml}


\subsection{Spring}\label{sec:traceAnalysis:instr:spring}

The Spring framework~\cite{Spring-WebSite} provides interfaces for intercepting %
Spring services and web requests. %
\Kieker{} includes the probes %
\class{OperationExecutionMethodInvocationInterceptor} and
\class{OperationExecutionWebRequestRegistrationInterceptor}. %
The \class{OperationExecutionMethodInvocationInterceptor} is similar to the %
AspectJ-based probes described in the previous section and monitors method %
executions as well as corresponding trace and session information. %
The \class{OperationExecutionWebRequestRegistrationInterceptor} intercepts %
incoming Web requests and initializes the trace and session data for this %
trace. If you are not using the \class{OperationExecutionWebRequestRegistrationInterceptor}, %
you should use one of the previously described Servlet filters to register %
session information for incoming requests %
(Section~\ref{sec:traceAnalysis:instr:servlet}).

See the Spring documentation for instructions how to add the interceptors %
to the server configuration.

\subsection{CXF SOAP Interceptors}\label{sec:traceAnalysis:instr:cxf}

The Apache~CXF framework~\cite{CXF-WebSite} allows to implement interceptors for web service calls, %
for example, based on the SOAP web service protocol. %
\Kieker{} includes the probes %
\class{OperationExecutionSOAPRequestOutInterceptor}, %
\class{OperationExecutionSOAPRequestInInterceptor}, %
\class{OperationExecutionSOAPResponseOutInterceptor}, and %
\class{OperationExecutionSOAPResponseInInterceptor} which can be used to %
monitor SOAP-based web service calls. %
Session and trace information is written to and read from the SOAP header of %
service requests and responses allowing to monitor distributed traces. %
See the CXF documentation for instructions how to add the interceptors %
to the server configuration.

\pagebreak

\section{Trace Analysis and Visualization}\label{sec:traceAnalysisTool}

\enlargethispage{0.5cm}

Monitoring data including trace information can be analyzed and visualized with the \KiekerTraceAnalysis{} tool which is included in the \Kieker{} binary as well.\\

\WARNBOX{
In order to use this tool, it is necessary to install two third-party programs:
\begin{enumerate}
\item \textbf{Graphviz} A graph visualization software which can be downloaded from \url{http://www.graphviz.org/}.
\item \textbf{GNU PlotUtils} A set of tools for generating 2D plot graphics which can be downloaded from \url{http://www.gnu.org/software/plotutils/} (for Linux) and from \url{http://gnuwin32.sourceforge.net/packages/plotutils.htm} (for Windows).
\item \textbf{ps2pdf} The \file{ps2pdf} tool is used to convert ps files to pdf files.
\end{enumerate}
Under Windows it is recommended to add the \dir{bin/} directories of both tools to the ``path'' environment variable. It is also possible that the GNU PlotUtils are unable to process sequence diagrams. In this case it is recommended to use the Cygwin port of PlotUtils.
}

\vspace{1mm}

\noindent Once both programs have been installed, the \KiekerTraceAnalysis{} tool can be used. It can be accessed via the wrapper-script \file{trace-analysis.sh} or \file{trace-analysis.bat} (Windows) in the \dir{bin/} directory. Non-parameterized calls of the scripts print all possible options on the screen. %, as listed in Appendix~\ref{appendix:wrapperScripts:traceAnalysis}.

The commands shown in Listings~\ref{lst:traceAnalysis:sequenceDiagram} and \ref{lst:traceAnalysis:sequenceDiagramWin} generate a sequence diagram as well as a call tree to an existing directory named \dir{out/}. The monitoring data is assumed to be located in the directory \dir{/tmp/kieker-20110428-142829399-UTC-Kaapstad-KIEKER/} or \dir{\%temp\%$\backslash{}$kieker-20100813-121041532-UTC-virus-KIEKER} under Windows. %

\setBashListing
\begin{lstlisting}[caption=Commands to produce the diagrams under \UnixLikeSystems,label=lst:traceAnalysis:sequenceDiagram]
#\lstshellprompt{}# #\textbf{./trace-analysis.sh}# #\textbf{--inputdirs}# /tmp/kieker-20110428-142829399-UTC-Kaapstad-KIEKER
                     #\textbf{--outputdir}# out/
                     #\textbf{--plot-Deployment-Sequence-Diagrams}#
                     #\textbf{--plot-Call-Trees}#
		     #\textbf{--short-labels}#
\end{lstlisting}

\begin{lstlisting}[caption=Commands to produce the diagrams under Windows,label=lst:traceAnalysis:sequenceDiagramWin]
#\lstshellprompt{}# #\textbf{trace-analysis.bat}# #\textbf{--inputdirs}# %temp%\kieker-20100813-121041532-UTC-virus-KIEKER
                    #\textbf{--outputdir}# out\
                    #\textbf{--plot-Deployment-Sequence-Diagrams}#
                    #\textbf{--plot-Call-Trees}#
		    #\textbf{--short-labels}#
\end{lstlisting}


\enlargethispage{1cm}

\WARNBOX{%
The Windows \file{.bat} wrapper scripts (including \file{trace-analysis.bat}) must be executed from within %
the \dir{bin/} directory.
}

\pagebreak

The resulting contents of the \dir{out/} directory should be similar to %
the following tree:

\begin{figure}[H]
\begin{graybox}
\dirtree{%
.1 \DirInDirTree{out/}.
.2 deploymentSequenceDiagram-6120391893596504065.pic.
.2 callTree-6120391893596504065.dot.
.2 system-entities.html.
}
\end{graybox}
\end{figure}

\noindent The \file{.pic} and \file{.dot} files can be converted into other formats, %
such as \file{.pdf}, by using the \textit{Graphviz} and \textit{PlotUtils} tools %
\file{dot} and \file{pic2plot}. %
The following Listing~\ref{lst:traceAnalysis:convertDiagrams} demonstrates this. %

% The generated diagrams are shown in the following %
% Figures~\ref{fig:traceAnalysis:callTree} and~\ref{fig:traceAnalysis:allocSeqDiagr}.

\input{ch5-trace-analysis_Convert_Diagrams.inc}

% \begin{figure}[H]\centering
%   \subfigure[]{\label{fig:traceAnalysis:callTree}%
%   \includegraphics[height=0.4\textheight]{images/callTree}
%   }%
%   \subfigure[]{\label{fig:traceAnalysis:allocSeqDiagr}%
%   \includegraphics[height=0.4\textheight]{images/allocationSequenceDiagram}
%   }%
%
%   \caption{Call Tree~\subref{fig:traceAnalysis:callTree} and Allocation Sequence Diagram~\subref{fig:traceAnalysis:allocSeqDiagr}}
% \end{figure}

\NOTIFYBOX{The scripts \file{dotPic-fileConverter.sh} and \file{dotPic-fileConverter.bat} %
convert all \file{.pic} and \file{.dot} in a specified directory. %
%See Appendix~\ref{appendix:wrapperScripts:dotPicFileConverter} for details.
}

\vspace{5mm}

Examples of all available visualization are presented in the following %
Section~\ref{sec:traceAnalysisExamples}.

\pagebreak

\section{Example \KiekerTraceAnalysis{} Outputs}\label{sec:traceAnalysisExamples}
\input{ch5-trace-analysis-exampleOutputs.inc}
